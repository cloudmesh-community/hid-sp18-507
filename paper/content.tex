% status: 100
% chapter: Blockchain

\title{Openchain}


\author{Stephen Giuliani}
\affiliation{%
  \institution{Indiana University}
  \state{Virginia}
  \country{USA}
}
\email{sgiulian@umail.iu.edu}

\renewcommand{\shortauthors}{S. Giuliani}

\begin{abstract}
Openchain is an open source blockchain technology developed so that enterprises could stand up their own, fully functional, secure, highly customizable, blockchain environment while maintaining full administration rights beginning with the transactions recorded on the chain themselves through the validation processes and even how the data is stored and used.
\end{abstract}

\keywords{hid-sp18-507, Openchain, Blockchain, Bitcoin, Open Source}

\maketitle

\section{Introduction}
Openchain is a blockchain technology designed as a simple yet customizable, enterprise-ready, cryptographic ledger to track the ownership of digital and real-world assets. Openchain offers both the back-end blockchain framework as well as a blockchain-enabled wallet for managing digital assets. The open source technology company leverages its online development community and offers support in designing and building a blockchain network for client-specific needs. Openchain's design differs from other and more traditional blockchain concepts by utilizing a client-designed hierarchy for validating assets and transactions, rather than using an anonymous decentralized validation system. Openchain's blockchain concept can be stood-up in seconds and is fully customizable for any individual's, company's, or asset-type's needs and the open-source nature ensures constant development and improvements.

\section{Blockchain}
In recent years, the concept of blockchain has established itself as a promising innovative technology with profitable, real-world implications and interest in blockchain technologies peaked in late 2017. But what is `Blockchain' and what does this technology mean for existing industries? In order to understand the capabilities and near-future potential impacts this technology will have on the world, we mist first understand what it is. Running a basic search on the web for `Blockchain' is sure to turn up resulting descriptions that use the terms `distributed ledger', `decentralized`, and `transparent'. Blockchain is a distributed ledger meaning that the data records (ledger) is duplicated and maintained on any number of computers/servers worldwide (distributed). The idea behind the distributed ledger is security in the sense that the data records must be identical across all servers so that any discrepancies, errors, or attempts to change information will be virtually impossible to occur. The concept of verifying information and ensuring no-foul play or errors occurred exists in many industries today but the process of guaranteeing the information is accurate is precisely what Blockchain technology is designed to tackle. A prime example is how banking institutions verify account balances and the ability to transfer money or pay for goods and services. A bank would have to verify the originating account's balance to ensure sufficient funds are available prior to the transaction. Then the bank would verify the address or location of the destination account; sometimes, this is a different banking institution with similar processes. Finally, once the transaction is completed, all parties must verify completion and maintain records of that transaction for any auditing purposes. Blockchain is expected to disrupt this process, and many similar in concept, by cutting out the middle man, or in this case, the banks. through the use of a blockchain, the account balance, address, destination address and balance are all connected on the same chain through information in some form or another. This information is stored as blocks, or essentially a collective of information making up some defined amount of memory, which are all connected (block-chain) and therefore can be verified through a query rather than (sometimes multiple) a middle-man like a bank. Financial transactions and institutions are not the only industry threatened by the use of blockchain technology. FUTURETHINKERS, a popular podcast covering evolving technologies, discusses 19 major industries that will likely be impacted by blockchain technology in a big way~\cite{FutureThinkers}. Any institution who's responsibility is to verify or maintain ownership of an asset can be disrupted by the adoption of blockchain; however many of these institutions are recognizing the potential of blockchain technology and have begun investing in utilizing the technology themselves. 

Understanding the distributed-ledger concept is only half of the defining capability within a blockchain. Blockchains are, by design, secure through the use of various cryptography practices. Blockchain technology utilizes cryptography and dual-pair public and private keys to verify the users of the chain itself. The information stored within the blocks of a blockchain is transparent so that anyone with access to the chain can view the information stored on the chain; however only the owner of the private key can make changes (depending on the chain, such as a transaction) to their account. This description is very high-level and the blockchain technologies in use today invoke a wide variety of characteristics that make each unique. Blocks can vary in size, enabling less or more information stored per block and entry within a block. The ability to view, add, or change the information within a blockchain can be enabled or disabled by the administer of the blockchain. And the purpose of the blockchain can vary from any need to store information securely.

Most notably in today's media, cryptocurrency is at the forefront of blockchain use and is the driving force behind its adoption as well as its hype. According to Google trends, the idea of cryptocurrency spiked interest far beyond blockchain, the technology behind cryptocurrency, tokens, and initial coin offerings, or ICOs (see Figure~\ref{f:googletrendchart}). 
\begin{figure}[!ht]
  \centering\includegraphics[width=\columnwidth]{../images/trendscryptoblockchain.png}
  \caption{Google Trends: Cryptocurrency vs. Blockchain~\cite{GoogleTrendsCrypto-Blockchain}}\label{f:googletrendchart}
\end{figure}
The hype for cryptocurrency and blockchain technologies has led to major jumps in various company stock prices who wither mention blockchain in a press release or even toy with the name ``Bit'' or ``Block'' as a part of the company name itself~\cite{ReutersKodak}. According to Gartner's Hype Cycle for Emerging Technologies for 2017, blockchain is well within the ``Peak of Inflated Technologies'' phase and therefore it will be some time before blockchain is adopted as the game-changer technology behind the hype~\cite{GartnerHypeTechnology2017}. More information on the technology behind and concept of Blockchain can be found at \textit{\url{https://www.ibm.com/developerworks/cloud/library/cl-blockchain-basics-intro-bluemix-trs/}}. Additionally, see Figure~\ref{f:blockchainpng} for a basic example of a transaction on a blockchain.

\begin{figure}[!ht]
  \centering\includegraphics[width=\columnwidth]{../images/blockchainbasics.png}
  \caption{Example Transaction on a Blockchain~\cite{BlockchainImage}}\label{f:blockchainpng}
\end{figure}

\section{Openchain}
Openchain aims to narrow the time to enlightenment in using blockchain and the Openchain technology enables developers, clients, consumers, and large businesses to deploy and benefit from the open source blockchain capabilities that are available already. Blockchain itself is traditionally implemented as a distributed ledger where each node or entity on the connected network keeps a copy of the ledger and through cryptography, anonymity, and the chain itself, guarantees security and reliability in maintaining accurate, valid ledger entries. One of the original and most famous implementations of the blockchain technology is Bitcoin, a decentralized digital currency with a price determined by the market for the currency itself. Bitcoin transactions, or the passing of the digital coin from one owner to another is verified and maintained by the blockchain so that no one ledger can claim  a false transaction compared to the other ledgers on the chain. Early adopters of blockchain usages boast that in order to fake a transaction or alter a transaction, the entirety of the chain would need to be reverified and re-recorded; this implies that thousands of nodes on a network would need to be changed instantly (a feat Don Tapscott, a Co-Founder of the Blockchain Research Institue once described as ``turning a Chicken McNugget back into a chicken''~\cite{DonTapscottPBS}), thus guaranteeing security. However, this concept of wide distribution, although secure, is not beneficial in a large-scaled environment. Bitcoin and other instantiations of this early concept of peer-to-peer, distributed, anonymous validation processes are often time consuming with verification times spanning multiple days while the nodes sync in validation. Challenges such as this can be solved by alternative blockchain technologies, such as the enterprise-ready blockchain technology by Openchain~\cite{CoindeskCoinprism}.

Coinprism, the parent company to Openchain, and its founder, Flavien Charlon, have pioneered multiple technologies utilizing blockchain concepts and developed Openchain as a way for large industries to tailor a blockchain to its own industry standards~\cite{BitcoinNewsOpenchain}. In a presentation with CoinDesk, a leading cryptocurrency and blockchain news outlet, Charlon explained that large industries and private companies saw the value in using a blockchain to manage asset ownership but were hesitant to adopt blockchain because of the increasingly longer validation cycles and veil behind not knowing who is a part of validating or denying transactions on an open chain~\cite{CoinDeskCharlonYouTube}. What makes Openchain a viable solution to private industry concerns, beyond its ability to be deployed in seconds, is that the private entity is the owner and administrator of all aspects of its own blockchain ledger. This means that a company who manages the land leases of thousands of lease contracts can enable its own third party resources to validate contracts rather than relying on the potential for lease-expertise or even competitors to approve transactions.

Openchain can function as a standalone blockchain, can be deployed beside other closed blockchains, or be developed as a ``sidechain'' to other existing blockchain technologies, such as a Bitcoin or Ethereum chain~\cite{OpenchainHome}. The end-users of a chain can exchange the asset freely within the openchain  per the rule and validation characteristics that the administrator sets. Administrators can elect to have asset-specific validating entities and can maintain multiple chains for each asset-type it handles. For instance, the company that maintains a ledger for land-leases can also administrate a ledger for the ownership of cars, gift-cards, medical expense history, or even digital currencies, and can choose to combine the chains to leave as individual standalone instances.

In addition to the blockchain technology designed for enterprise asset management, Openchain offers a digital wallet, compatible with its openchain technology so that the end-user can store the digitized asset and have any transactions and ownership validated by the chain its connected to. Through Openchain, a company, large or small, can implement and administer its own private blockchain as well as provide a digital wallet for client use. This model is common among small companies which use blockchain technologies and private coins or tokens to fund projects as a means of venture capitalism instead of the traditional and policy-laden concept of stocks through public offerings.

\section{Under The Hood}
Under the hood of an Openchain server you'll see Openchain uses a Docker image to set up the requirements and dependencies when spinning up the service. You can build an Openchain instance without docker and the documentation is provided by Openchain, although the configuration of both the docker-provided setup and the self instantiation are identical. The server itself defaults to using SQLite for the storage of transaction data, although you can use SQLServer. Additionally, customization and support for development and implementation are available via the active Openchain forum~\cite{OpenchainForum}. Openchain is written in JavaScript and therefore natively uses JSON formats for its HTTP API. The available methods include POST, for submitting a transaction for validation, and GET, for retrieving information of a validated transaction. Note, once a transaction is validated, there is no way to append or adjust the ledger entry. The validation process itself, via a POST method, involves submitting transaction data, which is mutated via a hex-encoded byte string, and then hashing the mutation via double SHA256~\cite{SHA256Wiki}. Finally, the transaction is signed using a private-public key pair, of which the private key is signed using Secp256k1, a popular cryptographic curve in blockchain signatures~\cite{Secp256k1Wiki}.

\section{Conclusion}
Despite the hype and the general population's misconceptions of blockchain, Openchain offers a simple, yet powerful opportunity for organizations and individuals to implement their own secure blockchain instance. Whether the purpose is to handle the initial offering of a (hopefully) useful investment currency, to backup personal digital assets, or to have a tangible back-story to claims that your company is exploring blockchain technology so that your public stock price soars, Openchain is available as an open source, highly customizable, simple, and widely supported blockchain technology.

\bibliographystyle{ACM-Reference-Format}
\bibliography{report}
